
\chapter{Nier:Automata}
\section{Der Hintergrund}
Nier:Automata ist ein Action basiertes Rollenspiel, welches am 23.02.2017 aus dem bekannten japanischem Studio Square Enix veröffentlicht wurde. Erhältlich ist es für PlayStation4, Windows (PC), und die XBox-One. Folglich deckt es fast den gesamten Markt ab und erfreut sich extrem großer Spielerzahlen. Aufgrund eines hohen Gewaltpotentials sowie recht schwer zu verstehenden Dialogen, ist das Spiel in Deutschland mit FSK 16 an den Start gegangen. Der 2te Teil von Nier (Nier:Automata) spielt im selben Universum wie sein Vorgänger Nier. Das Spiel erreichte sehr gute Kritiken bei diversen Plattformen. Im Vordergrund stehen die menschenähnlichen humanoiden Roboter welche zwar ihren Aufgaben nachgehen und dennoch immer mehr menschliche Emotionen entwickeln.

  \begin{quote}
	Die Figuren, die nicht müde werden, ihre Künstlichkeit zu betonen, dabei aber mehr menschliche Züge zeigen als ihnen lieb sein kann, sorgen dafür, dass ich mit ihnen leide, mit ihnen hasse und mit ihnen hoffe. (Quelle: 4players.de)
	\end{quote}

So schafft Taro Yoko ein weiteres starkes Abenteuer im Nier beziehungsweise Drakengard Stile. Durch tiefe Emotionen wird der Spieler an das Spiel gebunden. Jedoch wird durch actiongeladene Kämpfe, ein sehr gut balanciertes wie auch verstörendes Spiel geschaffen. 
Durch das vorhandensein von insgesamt 26 möglichen Enden ist ein sehr langer Spielspaß garantiert.

\section{Die Hauptcharaktere}

\subsection{Die Androidin 2B}

\begin{quote}
	\dq{Everything that lives is designed to end. We are perpetually trapped...in a never-ending spiral of life and death.}
	(Quelle: 2B in der Prolog Sequenz)
\end{quote}
2B ist die Hauptfigur in der Nier Reihe. Sowohl im ersten wie auch im zweiten Teil ist sie die tragende Figur. Sie ist wie alle anderen YoRHa-Mitglieder eine Androidin, welche mit 9s die Erde von Maschinen befreien muss. Sie soll die Welt von allen "nicht so weit entwickelten" Maschinen befreien und sie so für die Rückkehr der Menschen auf die Erde vorbereiten. Diese leben aktuell auf einer Station auf der Rückseite des Mondes.  2B ist eine sehr pflichtbewusste Androidin. Sie stellt ihre Befehle über alles andere. Sie redet mit anderen Charakteren nicht mehr als wirklich notwendig ist. Ab und zu wirkt sie sehr sardonisch, was sie recht schnell hitzköpfig darstellen lässt. Emotional ist 2B recht kalt. Das ändert sich, je länger sie mit 9S zusammen arbeitet. Anfangs versucht 2B keine Emotionen zuzulassen aber mit 9s Hilfe schafft sie es, diese zuzulassen. Es entwickelt sich über die Zeit eine tiefe Beziehung zu ihm, welche 2B niemals zugeben würde. Dennoch wird diese sehr stark im Kampf gegen \dq Adam und Eva \dq  deutlich denn sie zeigt Eve gegenüber grenzenlosen Hass. Durch den tiefen Konflikt zwischen ihrem Pflichtbewusstsein und den Gefühlen zu 9S wird sie immer mehr hin und hergerissen und entscheidet sich schlussendlich (je nach Ende) für 9S. ~\cite{nier:automatawikia20192b}


\subsection{Der Android 9S}

\begin{quote}
	\dq I‘m not quite sure what it means to mourn, or even if we have a soul to concern ourselves with… But I hope you’re at rest 2B. Sweet dreams. I’ll be with you before long.\dq  \vspace{10px}(9S am Ende B)
\end{quote}
	9S ist im Gegensatz zu 2B nicht für den Kampf vorgesehen. Er ist ein Scanner Android und trägt somit die Aufgabe Informationen und Kartenmaterial zu besorgen. Trotz seiner \dq Einfachheit \dq entwickelt er schneller als 2B Emotionen die Abseits seiner Programmierung liegen. Er ist in fast allen Enden, abgesehen von B, C und D an der Seite von 2B und unterstützt diese. Er ist immer höflich und zuvorkommend. Das spiegelt sich auch in der Anrede von 2B mit \dq Ma'am \dq wieder . Das besondere an 9S ist, das er immer fröhlich und optimistisch ist, wenn er auch gegen Ende seine Emotionen mit 2B zu tauschen scheint. Wie es seiner Programmierung entspricht, ist er ein sehr nützlicher \dq Helfer \dq . 9S Forscherdrang und sein Wunsch, das große Ganze zu sehen, stehen ihm sehr oft im Weg. Leider wird er in fast allen Enden wahnsinnig und 2B ist verpflichtet ihn ziemlich oft zu töten. ~\cite{nier:automatawikia20199S}

\section{Die wichtigsten Nebencharaktere}


\subsection{Adam und Eva}
\begin{quote}
	\dq I—or we machine lifeforms I suppose—have a keen interest in humanity. Love. Family. Religion. War. The more human records I unearth, the more charmed I am by their complexity. This city is one of many areas I've created out of a desire to understand...to know...humans. It's grand, don't you think? Almost...spiritual. And yet, it's nothing more than an android graveyard. I seek to learn and and adopt all facets of humanity! Some desire love! Others family! Only then did I realize the truth...the core of humanity...is conflict. They fight. Steal. Kill. This is humanity in its purest form! \dq (Quelle: Adam zu 2B in der Stadt der Erinnerungen) ~\cite{verschiedene2019npc}
\end{quote}

Adam und sein jüngerer Bruder Eva sind ebenfalls humanoide Androiden. Sie wurden zwischen dem 10. März und dem 7. April 11945 geboren. Folglich sind beide unter einem Jahr alt. Über Eve ist nicht wirklich etwas bekannt. Über Adam hingegen weiss man nur, dass er ziemlich kühl und Herrschsüchtig ist. Abgesehen von den beiden Kampfsequenzen sieht man die beiden meist an einem  Tisch sitzend, ähnlich dem des letzten Abendmahles. Adam ist sehr an den Menschen interessiert und versucht alles, um selber so nah wie möglich selbst einer zu werden. Vor allem seine Expertise über den menschlichen Tot fasziniert ihn. Er versucht ihn oft nach zuempfinden und zu verstehen. Jedoch gelingt ihn das nicht, da er alle \dq Wunden \dq einfach regeneriert. Beide versuchen, wenn auch Eve eher abgelehnt wirkt, trotz ihrer androiden Bauform, Menschen zu werden. Folglich fügt er sich auch selber Schmerzen zu, da Schmerzen auch ein Teil des Menschsein sind. Dennoch wirkt er viel komplexer, tiefer und trauriger als Eve. Er eifert den Menschen nach, ohne die Möglichkeit zu haben, selbst einer zu werden.

\clearpage


\subsection{Die Androidin 2A}
\begin{quote}
	\dq I never quite realized ... how beautiful this world is.\dq \\ (Quelle: 2A am Ende C)
\end{quote}
Auch 2A diente im 14. Maschinenkrieg. Sie ist eine sehr aufbrausende Figur, auch wenn sie sich oft zurückhält und kaum redet. Durch ihre innige Leidenschaft zu kämpfen, gerät sie immer wieder zwischen die Fronten. 2A bringt sich durch unüberlegte und rücksichtlose Handlungen stetig in Gefahr. Als ihr gesamtes Cluster gefallen ist, verliert sie jeglichen Verstand und ihr Verlangen nach Zerstörung allen Lebens nimmt überhand. Trotz allem ist sie nicht ganz ohne Mitgefühl, so hilft sie die Maschinenkinder aus der Fabrik zu retten und tötet 2B auf ihren Wunsch hin. 2A ist einer der widersprüchlichsten Charaktere im gesamtem Spiel. Trotz ihres grenzenlosen Hasses auf alle Maschinen und sonstige Lebensformen, welche sie auch ohne Rückhalt äußert, hilft sie Schwächeren und folgt den Bitten von 2B. 9S hofft bis zum Rindringen in den Marmorturm, das sich 2A von selbst auf ihre eigenen Werte beruft und wieder eine normale humanoide Androidin wird und zu den YoraH Streitkräften zurückkehrt.
 

\section{Die Story}

\subsection{Das Intro}
In dem Intro lernt der Player das Spiel und die Steuerung der Androiden sowie die der Flugzeuge kennen. 2B landet in einer Fabrik wo sie mit Hilfe von 9S eine Maschine der Goliath Klasse zerstören soll. Diese Wesen sind ähnlich wie die Yohrah Androiden mechanisch, jedoch fehlt ihnen jegliche Möglichkeit Empathie zu empfinden. Sie schaden sich und der Umwelt und müssen daher beseitigt werden.

\subsection{Die Mainline}
Das Spiel handelt in einer sehr fernen Zukunft. Die restlichen Menschen wohnen angeblich in einer Kolonisation auf der Rückseite des Mondes. Die Androiden haben nun die Aufgabe die Erde von \dq schlechten, emotionslosen Schrottmonstern \dq zu säubern. Folglich, von allem was nicht menschlich und nicht sie selber sind. Im Laufe des Spieles, treffen die beiden Protagonisten jedoch auf verschiedene Personen beziehungsweise Charaktere, die Zweifel an diesem Leben haben. So wird die kalte und berechenbare 2B immer menschlicher zweifelt an ihren Aufgaben. Bei 9S geht dieser Prozess deutlich schneller. Sie erkennen, dass auch die in den Yorah Augen als feindlichen Goliath Wesen ein eigenes Leben haben. Sie erkennen, das nicht alle böse sind und manche sogar eine eigene Zivilisation aufgebaut haben, in welcher sie die Menschen imitieren. Nun stehen 9S und 2B im Zwiespalt mit dem emotionsbehafteten Maschinen und ihrem Auftraggeber. 

\subsection{Das Ende}
Gegen Ende lernen die Protagonisten die Charaktere Adam und Eva kennen. In der englischen Version heißt sie Eve. Zur Vereinfachung, wird er einfach Eva genannt. Beide eifern den Menschen nach und wollen um jeden Preis welche sein. Auch dann, oder gerade weil sie das Konzept von Tod und Schmerz nicht verstanden haben. 2B und 9S bekämpfen die beiden und gehen schließlich als Sieger hervor. Jedoch sind beide schwer verletzt. Ein einzelnes Ende gibt es nicht. Je nachdem, wie sich der Spieler entscheidet, stehen ihm 26 verschiedene Enden zur Verfügung. 