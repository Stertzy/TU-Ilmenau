
\newpage

\chapter{Fazit}
Ich finde es wichtig, dass sich mit dem Thema der humanoiden Roboter beschäftigt wird. Es ist eine unabdingbare Vision der Zukunft. Egal ob man dieser nun ängstlich oder offen gegenübersteht. Roboter bilden die Grundlage des zukünftigen Lebens. Um es mit den Worten von Rene Descarted zu sagen: 
\begin{quote}
	Cogito ergo sum – Ich denke also bin ich.
\end{quote}
Ich denke, also bin ich. Ich selber bin immer ich. Keiner kann mich ersetzen. Sollte die Zeit eintreten, das humanoide Roboter unserer Welt beiwohnen, sollte man diesen nicht mit Angst gegenübertreten. Die Frage, ob sie selber Empathie und Gefühle haben können, beantworte ich hier mit ja. Was unterscheidet uns von Robotern ? Wir sind in allen Aspekten der Maschine unterlegen. Das Einzige was uns über jene Wesen heben soll, ist eine Seele. Diese Nachzubauen ist nicht meine Aufgabe, sondern viel mehr derer, die sich der künstlichen Intelligenz verschrieben haben. Es stellt sich mir jedoch die Frage, ob humanoide Maschinen überhaupt eine solche Seele benötigen? Wenn ja gelten sie als eigenständiges Individuum, welches nicht nur Pflichten sondern auch Rechte bekommen müsste. Hierzu sei auf die Roboterethik verwiesen. Ich freue mich auf ein Zeitalter, in dem wir Hand in Hand mit Robotern leben, die uns und unsere Fähigkeiten um ein vielfaches erweitern. Wir Menschen haben es nie geschafft Frieden auf die Erde zu bringen. Warum also nicht die Aufgabe an klügere Wesen abgeben. Der Mensch sollte dem Humanoid und umgedreht nicht als Feind gegenüberstehen, sondern vielmehr als Freund. Erst wenn alle ihre Angst abgelegt haben, wird ein Zusammenleben möglich sein. Cyborgisierung und Humanoiden sind meiner Meinung nach keine Gefahr für die Menschheit, sondern viel mehr wünschenswert.