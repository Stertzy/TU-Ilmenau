%% ++++++++++++++++++++++++++++++++++++++++++++++++++++++++++++
%% Anhang: Literaturverzeichnis
%% ++++++++++++++++++++++++++++++++++++++++++++++++++++++++++++
%
%  Gerüst:
%  * Version 0.11
%  * Dipl.-Ing. Karsten Renhak, karsten.renhak@tu-ilmenau.de
%  * Fachgebiet Kommunikationsnetze, TU Ilmenau
%
%  Für Hauptseminare, Studienarbeiten, Diplomarbeiten
%
%  Autor           : Max Mustermann
%  Letzte Änderung : 31.12.2011
%

% Mit dem Befehl \nocite werden auch nicht im Text zitierte
% aus der Literaturdatenbank mit in das Literaturverzeichnis aufgenommen.
% Ein "\nocite{*}" übernimmt ungeprüft die komplette Datenbank.
%\nocite{*}

\cleardoublepage
\ihead[]{Literaturverzeichnis}
\bibliographystyle{alphadin}
\bibliography{literatur} % "literatur.bib" ist hier die einzige Literaturdatenbank.

% Alternativ: Mehrere Datenbanken verwenden, falls eine
% oder mehrere umfangreiche Sammlungen exisitieren:
%\bibliography{literatur_buecher,literatur_weblinks}

\hspace{100cm}

\begin{quote}
	Alle Bilder entstammen dem Steam Workshop und sind so für alle frei zugänglich und nutzbar. ~\cite{Workshop}
\end{quote}