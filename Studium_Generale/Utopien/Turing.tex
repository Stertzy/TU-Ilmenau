\chapter{Der Turing Test}

\section{Alan Turing}
Alan Mathison Turing wurde am 23. Juni 1912 in London geboren. Er wirkte als Kryptologe, Mathematiker und Informatiker. Turing war an vielen technischen Entwicklungen seiner Zeit beteiligt, so unter anderem auch an der nach ihm benannten Turing Maschine. Diese modelliert auf einfachste Weise die Arbeitsmethodik eines Computers. Eingeführt wurde sie bereits 1936. Zu diesem Zeitpunkt war Turing erst 24 Jahre alt. Als Brite war er unter anderem an der Dechiffrierung der deutschen Funksprüche beteiligt. Diese nutzten sie seit 1918. Im Jahr 1950 entwickelte Turing den bekannten Turing-Test. Durch diesen Test soll herausgefunden werden, ob ein Gegenüber ein Mensch oder ein Roboter ist. Wie dieser Test abläuft wird später genauer erklärt. Da Homosexualität unter Georg VI noch unter Strafe stand, wurde Turing 1952 verhaftet und zu einer chemischen Kastration verurteilt. Er starb daraufhin an  Depressionen. ~\cite{Turing_Leben}

\section{Der Turing Test}
Der Grundaufbau des Testes ist eigentlich recht simpel. Obwohl es mehrere Versionen und Arten gibt, bleibt die Grundidee immer gleich. Es gibt eine künstliche Intelligenz (welche es zu testen gilt), einen menschliche Intelligenz und einen menschlichen Prüfer. Der Prüfer hat keinen Kontakt zu den beiden Individuen. Ausschließlich über jeweils eine Tastatur kann er zu den Subjekten Kontakt aufnehmen und sich unterhalten. Heutzutage wird auch, dank der fortschreiten Technik, über eine Audiokommunikation nachgedacht. Der Prüfer unterhält sich nun mit beiden Individuen, um herauszufinden, bei welcher es sich um eine künstliche Intelligenz handelt und bei welcher es sich um einen Menschen handeln könnte. Kann dieser nicht eindeutig herausfinden wer die Maschine ist und wer der Mensch ist, gilt der Test als bestanden. ~\cite{gregorlüdi/martinlüscher20077}

\section{Kritik am Turing Test}
Mit der Zeit werden auch immer mehr Stimmen laut die den Turingtest für überholt und für zu fehlerhaft halten. Laut John Searle zeigt der Test legendlich die Funktionalität der KI auf. Jedoch nicht, dass die Intelligenz wirklich ein Bewusstsein entwickelt und auch nicht dass sie ihre Gedanken auf ein Ziel reflektieren kann. ~\cite{jaai2019}

\section{Alternativen}
Durch den Wandel der Technik bildeten sich auch andere Methoden zur Überprüfung von künstlicher Intelligenz heraus. Diese werden hier nur kurz aufgeführt. 
\begin{description}
	\item[Metzinger-Test]
		Die KI muss hier um den Test zu bestehen nicht nur logisch antworten, sondern auch eigene Thesen formulieren und diese dann auch sinnvoll vertreten. Diese müssen jedoch auch auf Grund der eigenen Fähigkeiten begründet und fundiert sein. 
	\item[Lovelace-Test] In diesem Test muss eine künstliche Maschine originäre Leistungen erbringen. Das heißt, sie muss etwas erschaffen, für das sie nicht programmiert wurde. 
\end{description}

