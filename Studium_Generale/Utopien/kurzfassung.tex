%% ++++++++++++++++++++++++++++++++++++++++++++++++++++++++++++
%% Zusammenfassung, Abstract
%% ++++++++++++++++++++++++++++++++++++++++++++++++++++++++++++
%
%  Gerüst:
%  * Version 0.10
%  * Skopp Jonathan, jonathan.skopp@gmail.com
%  * Fakultät IA, TU Ilmenau
%
%  Für Hauptseminare, Studienarbeiten, Diplomarbeiten, Studium Generale
%
%  Autor           : Jonathan Skopp
%  Letzte Änderung : 09.05.2019
%

\renewcommand{\abstractname}{Kurzfassung}
\begin{abstract}
\ldots \emph{Die vorliegende Hausarbeit beschäftigt sich mit der Dystopie der Transzedenz in Bezug auf das Action-Rollenspiel Nier:Automata. Dabei werden die Begriffe wie Dystopie und Transzendenz erklärt und sie in Bezug gesetzt. Die Nähe dieser Dystopie und der aktuelle Zustand werden erklärt und verdeutlicht. Können Roboter fühlen ? Sollten Sie fühlen und eigenständig denken können ? Ersetzen uns Maschinen in naher Zukunft? Kernthema der Arbeit ist die Dystopie in Bezug auf ein exemplarisches Beispiel unter Berücksichtigung des heutigen Standes der Entwicklung.}\ldots
\par{}




\end{abstract}

\renewcommand{\abstractname}{Abstract}
\begin{abstract}
\ldots \emph{This housework deals with the dystopia of transcedence in relation to the action role-playing game Nier:Automata. The terms dystopia and transcendence are explained and related. The proximity of this dystopia and the current state are explained and clarified. Can robots feel ? Should you be able to feel and think independently ? Will machines replace us in the near future? The core topic of the work is dystopia in relation to an exemplary example, taking into account the current state of development.}\ldots
\end{abstract}

