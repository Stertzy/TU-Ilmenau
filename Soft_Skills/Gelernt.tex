\chapter{Was habe ich gelernt}

Der große Unterschied zu anderen Modulen ist, dass hier praktischen Wissen vermittelt wird. In Mathematik oder Informatik lernt man wissen, welches man zwar anwenden kann aber kaum praktischen Nutzen hat.\\
Als Beispiel kann man hier sehr gut die Zwischenmenschliche Kommunikation anführen. In den Videos und dem Skript von Herrn Balkow konnte man Verhaltensmuster lernen, welche für das gesamte Leben nützlich sind. So erlernte ich dort welche verschiedenen Varianten eine gelungene Kommunikation herrühren. Die folgende Auflistung stammt aus dem Skript ~\cite{Skript}. Um diese hier einmal zu nennen:
\begin{itemize}
	\item[-] Beobachten
	\item[-] Zuhören
	\item[-] Wahrnehmen
	\item[-] Reflektieren
	\item[-] Lernen
\end{itemize} 
 
Das mag an einigen Stellen jetzt durchaus banal und einfach klingen. Jedoch gibt es selbst in der heutigen Zeit viele Probleme, welche man mit einer einfachen Unterhaltung hätte lösen können. Das schöne ist, das es mir möglich war mich privat weiter mit dem Thema auseinander zu setzten. So konnte ich mit ~\cite{Schlüssel} und ~\cite{fRede} 