\chapter{Vorwissen}

Im Vorfeld auf das Modul "Soft Skills" hatte ich noch nicht wirklich ein Vorwissen auf das Thema. Vorträge, Gesprächsrunden und ähnliches waren mir bis dahin sehr fremd. Um meine Meinung zu diesem Thema zu verbalisieren möchte ich gern ein Zitat von Madame Pompadour anfügen

\begin{quote}
	\dq{Man kann mit jedem Menschen reden. Die Kunst besteht darin, es zu vermeiden.} \dq\\
     Madame Pompadour
\end{quote}

Dennoch bildet der mündliche Austausch von Meinung über Sprache und Rede einen wichtigen Grundstein in der heutigen Gesellschaft.  \\

Doch was sind nun Soft Skills? 
Soft Skills bezeichnet nicht eine abgeschlossene Menge von Fähigkeiten sondern viel mehr einen Pool an verschiedenen Skills. Sie definiert sich über persönliche Eigenschaften (Geslassenheit, Gedult, Freundlichkeit), persönliche Werte (Fairness, Respekt) sowie über soziale Kompetenzen (Teamfähigkeit, Kommunikationsfähigkeit)  ~\cite{SoSk} \\

Dies war im Grunde das einzige was ich über das Thema wusste. Es spiegelt auch sehr gut meine anfängliche Meinung über das Modul dar. Letztere hat sich jedoch schnell geändert. In dem Modul sollten zwar Hauptsächlich die Zwischenmenschliche Kommunikationen eine Rolle spielen, aber auch andere persönlichen Skills.
Der große Unterschied war die Herangehensweise an das Modul. Normal lernt man das Wissen in Vorlesungen und übt dieses Wissen in Übungen. Hier konnten wir uns in Gruppen zusammensetzen und Aufgaben in Video Form lösen. 