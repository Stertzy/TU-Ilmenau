\subsection{Machbarkeitsanalyse}

\begin{itemize}
	\item Benutzerkennung
		\begin{itemize}
			\item Hier handelt es sich um Wunschfunktionen. Die Passwort Abfrage lässt sich realisieren und ist so als machbar anzusehen.
		\end{itemize}
	
	\item Kontakte hinzufügen
		\begin{itemize}
			\item Die Funktion mit einem QR Code lässt sich mithilfe der Spring 5 API recht gut und solide implementieren. Daher ist auch diese Funktion als machbar anzusehen.
		\end{itemize}
	
	\item Kontaktverwaltung
		\begin{itemize}
			\item Die Verwaltung der Kontakte erfolgt über eine Datenbank. In dieser ist es möglich, Kontakte anzulegen und zu entfernen. Diese Funktion ist als machbar anzusehen.
		\end{itemize}
	
	\item Chatfunktionen
	 \begin{itemize}
		\item Alle Funktionen sind machbar. 
	\end{itemize}
	
	\item Personalisierung
	\begin{itemize}
		\item Hierbei handelt es sich um triviale Funktionen der GUI. Da Kotlin CSS unterstützt ist diese Funktion durchaus machbar.
	\end{itemize}
	
	\item Hydra Funktionen
	\begin{itemize}
		\item Alle Hydra Funktionen sind möglich und umsetzbar. 
	\end{itemize}
	
	\item Fragmentierungsfunktion
	\begin{itemize}
		\item Hierbei handelt es sich um eine Wunschfunktion. Diese wird auch als solche behandelt. Es ist machbar, Nachrichten zu komprimieren und in Segmente zu zerlegen.
	\end{itemize} 

	\item Kommunikationsfunktion
	\begin{itemize}
		\item Sämtliche Pflichtfunktionen unterliegen dem generellen Ziel der App und sind machbar.
	\end{itemize}



\end{itemize}

