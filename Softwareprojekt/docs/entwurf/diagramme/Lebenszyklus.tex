\documentclass[a4paper]{article}

\usepackage[utf8]{inputenc}
\usepackage[margin=2cm]{geometry}

\title{Grobentwurf v0.2}
\author{Philipp Schock}
\date{\today}

\begin{document}
    \maketitle

    \section{Lebenszyklus einer Nachricht}
    \begin{enumerate}
        \item
            Nutzer tippt Nachricht in Chat-Activity ein.

        \item
            Nachricht wird in der Datenbank abgespeichert.

        \item
            Background-Service übergibt sie dem C++ Code.

        \item
            Die Nachricht wird komprimiert.

        \item
            Die Nachricht wird in der Cell-Send-Queue abgelegt.

        \item
            Die Nachricht wird Fragment für Fragment aus der Cell-Send-Queue genommen, Ende-zu-Ende verschlüsselt,
            in Circuit-Cells gepackt und onion-verschlüsselt.

        \item
            Die Cell wird versendet.

        \item
            Die Cell kommt beim anderen Nutzer an.

        \item
            Die onion- und Ende-zu-Ende-Verschlüsselung werden entfernt.

        \item
            Die Nachricht wird Fragment für Fragment in die Cell-Receive-Queue geschrieben, bis alle Fragmente angekommen sind.

        \item
            Die Nachricht wird dekomprimiert.
            
        \item
            Die Nachricht wird an den Background-Service übergeben.

        \item
            Der Background-Service speichert die Nachricht in der Datenbank.
            Er zeigt dem User eine Push-Benachrichtigung an.

        \item
            Die Activity liest die Nachricht aus und zeigt sie dem Nutzer an.
    \end{enumerate}
\end{document}
