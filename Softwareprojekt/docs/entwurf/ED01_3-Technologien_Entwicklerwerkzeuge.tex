\subsection{Technologien und Entwicklungswerkzeuge}

\subsubsection{Programmiersprachen}
\textnormal{Für die \ac{GUI} und den \ac{MS} wird \textbf{Kotlin} verwendet, eine bevorzugte Programmiersprache für Android-Apps. Für Fragmentierung und Komprimierung, die \ac{E2EE}, \ac{OE} und \ac{gRPC} wird \textbf{C++} verwendet, welche eine bevorzugte Programmiersprache für die effiziente Implementierung von Anforderungen ist.}

\subsubsection{Plug-Ins}
\textnormal{Es wird \textbf{\ac{NDK}} verwendet, um Teile des Codes in C++ schreiben zu können und \textbf{CMake} zum erstellen von Makefiles. Außerdem wird als API \textbf{\ac{JNI}} verwendet, um eine Schnittstelle zwischen dem Code in Kotlin und C++ zu ermöglichen.}

\subsubsection{Bibliotheken}
\textnormal{Es wird \textbf{libcrypto++} verwendet, um auf cryptografische Algorithmen zugreifen zu können. Außerdem kommt \textbf{\ac{ECDH}} zum Einsatz, um Schlüssel zwischen den Kommunikationspartnern auszutauschen. Zudem wird  \textbf{\ac{gRPC}} verwendet, um sowohl mit dem Entrymix, als auch mit dem \ac{DS} zu kommunizieren.}

\subsubsection{Tools}
\textnormal{Für die Versionierung, Wiki und das Bugtracking verwenden wir \textbf{GitLab}, welches auf Basis von Git arbeitet.}
