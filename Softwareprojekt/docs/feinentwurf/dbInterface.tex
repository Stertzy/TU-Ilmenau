\documentclass[a4paper]{article}

\usepackage[margin=2cm]{geometry}
\usepackage[utf8]{inputenc}

\title{Schnittstellendefinition der Datenbankkomponente}
\author{Philipp Schock, Lukas Meinl}
\date{\today}

\begin{document}
    \maketitle

    \section{Einleitung}

    Die Folgende Schnittstellendefinition ist die Menge der Funktionen/Methoden, die die Datenbankkomponente
    mindestens für die GUI und den Main-/Background-Service bereitstellen muss.
    Abhängig von der inneren Architektur und letztendlichen Implementierung können sie zum Beispiel als statische Funktionen,
    Methoden eines Sigletons oder Methoden einer Klasse von Objekten realisiert werden.
    Die Wahl sollte so getroffen werden, dass das \texttt{Room}-Objekt so selten wie möglich instantiiert
    (also so oft wie möglich wiederverwendet) wird, da dies sehr aufwendig ist.

    \section{Definition}

    \subsection{Zugriffe auf GUI-Seite}

    \subsubsection{\texttt{getContacts()}}
    Gibt alle Kontakte des Nutzers zurück.
    Je nach Implementierung kann dies als Liste/Array, Stream, Iterator o.ä. gemacht werden.
    Diese(r) muss sich \textbf{nicht} automatisch aktualisieren.
    Der hierfür verwendete Datentyp \texttt{Contact} wird im Anhang spezifiziert.

    \subsubsection{\texttt{getMessages(int contactID)}}
    Gibt alle Nachrichten des Nutzers mit ID \texttt{contactID} zurück.
    Je nach Implementierung kann dies als Liste/Array, Stream, Iterator o.ä. gemacht werden.
    Diese(r) muss sich \textbf{nicht} automatisch aktualisieren.
    Der hierfür verwendete Datentyp \texttt{Message} wird im Anhang spezifiziert.

    \subsubsection{\texttt{addMessageToSend(int contactID, String content)}}
    Schreibt eine neue Nachricht an den Kontakt mit der ID \texttt{contactID} in die Datenbank.
    Die restlichen Attribute der Nachricht müssen hier berechnet werden.

    \subsubsection{\texttt{addContact(String nickname)}}
    \textbf{Zu Testzwecken vereinfachte Implementierung:}
    Erstellt einen neuen Kontakt mit Anzeigenamen \texttt{nickname}.

    \subsubsection{\textit{Optional}: \texttt{setNickname(int contactID, String nickname)}}
    Ändert den Anzeigenamen des Kontakts mit der ID \texttt{contactID} in der Datenbank.

    \subsubsection{\textit{Optional}: \texttt{setTrusted(int contactID, boolean trusted)}}
    Ändert den Vertrauensstatus des Kontakts mit der ID \texttt{contactID} in der Datenbank.

    \subsection{Zugriffe auf Service-Seite}

    \subsubsection{\texttt{getUnsentMsg()}}
    Gibt eine Nachricht (\texttt{Message}) zurück, für die entweder
    \begin{itemize}
        \item
            \texttt{PassedToBackend?} falsch ist oder
        \item
            \texttt{TimeSent} länger als \texttt{timeout} (siehe Anhang) in der Vergangenheit liegt und \texttt{Acknowledged?} falsch ist.
    \end{itemize}
    \texttt{PassedToBackend?} soll hierbei auf wahr gesetzt werden.

    \textbf{Zu Testzwecken vereinfachte Implementierung:}
    nur erstes Kriterium.

    \subsubsection{\texttt{addReceivedMessage(Message msg)}}
    Schreibt eine neue Nachricht in die Datenbank.
    Die drei Attribute \texttt{PassedToBackend?}, \texttt{TimeSent} und \texttt{Acknowledged?} sollen auf \texttt{null} o.ä. gesetzt werden.
    Die restlichen Attribute der Nachricht müssen hier berechnet werden.

    \subsubsection{\textit{Optional}: \texttt{setSent(int contactID, int msgID)}}
    Setzt ausgehende Nachricht, die auf \texttt{contactID} und \texttt{msgID} passt, als gesendet.
    Die Zeit muss hier berechnet und geschrieben werden.

    \subsubsection{\textit{Optional}: \texttt{setAcknowledged(int contactID, int msgID)}}
    Setzt ausgehende Nachricht, die auf \texttt{contactID} und \texttt{msgID} passt, als acknowledged.

    \section{Anhang}

    \subsection{Datentyp für Kontakte}

    Der \texttt{Contact}-Datentyp sollte folgende Attribute haben, sowie \texttt{get} und \texttt{set} Methoden für diese:
    \begin{itemize}
        \item
            \texttt{id:int}
        \item
            \texttt{nickname:String}
        \item
            \texttt{trusted:boolean}
        \item
            \texttt{hasChat:boolean}
    \end{itemize}

    \subsection{Datentyp für Nachrichten}

    Der \texttt{Message}-Datentyp sollte folgende Attribute haben, sowie \texttt{get} und \texttt{set} Methoden für diese:
    \begin{itemize}
        \item
            \texttt{contactID:int}
        \item
            \texttt{msgID:int}
        \item
            \texttt{received:boolean}
        \item
            \texttt{content:String}
        \item
            \texttt{sent:boolean}
        \item
            \texttt{acknowledged:boolean}
    \end{itemize}

    \subsection{\textit{Optional}: Timeout}

    Der Timeout soll vorerst als Konstante in der Datenbankkomponente implementiert werden.
\end{document}
