\section{Meilensteine}
\label{sec:Meilensteine}
Im Folgenden werden die wichtigsten Meilensteine unserer App-Entwicklung aufgezeigt.

\begin{itemize}
	\item Bis 13.05.2020:
	\begin{itemize}
		\item Pflichtenheft fertig
		\item Organisationsdokument fertig
		\item Grobentwurf fertig
	\end{itemize}
	
	\item Bis 21.05.2020:
	\begin{itemize}
		\item Provisorische Background-Services
		\item GUI Feinentwurf
		\item Zusammenstellen der Kryptographie-Bibliothek (sodass alle kryptographischen Hilfsmittel, die für den Client benötigt werden, zur Verfügung stehen)
	\end{itemize}
		
	\item Bis 28.05.2020:
	\begin{itemize}
		\item Provisorische GUI mit allen Schaltflächen der Pflichtfunktionen zur
		Nutzerinteraktion (Funktionalität dieser Schaltflächen ist nicht zwingend gegeben zu diesem Zeitpunkt)
		\item DB steht und ist funktionsfähig (siehe Entwurfsdokumentation 2.2.1)
		\item Einfacher Prototyp zum testweise Versenden einer Nachricht von Alice an Bob
	\end{itemize}		
	
	\item Bis 04.06.2020:
	\begin{itemize}
		\item Directory fetch (der Client ist in der Lage alle benötigten Informationen vom Directory Service zu holen und diese für die weitere Verarbeitung nutzbar abzuspeichern)
		\item Setup-Pakete werden nach den Hydra-Spezifikationen erstellt und versendet
		\item Acknowledgements werden korrekt generiert/verarbeitet und an den
		richtigen Empfänger versendet
	\end{itemize}		
	
	\item Bis 10.06.2020:
	\begin{itemize}
		\item Das Komprimieren von Nachrichten funktioniert
		\item Alle Pflichtfunktionen wurden implementiert
	\end{itemize}	
	
	\item Bis 18.06.2020:
	\begin{itemize}
		\item Alle bereits bekannten Fehler beheben
		\item Kernfunktionen testen
	\end{itemize}
	\item Bis 25.06.2020:
	\begin{itemize}
		\item Alle Pflichtfunktionen testen
		\item Dabei auftretende Fehler beheben
	\end{itemize}
	\item Bis 02.07.2020:
	\begin{itemize}
	\item Das Gesamtsystem automatisiert auf vollständige Funktionalität überprüfen.
		\item Besonders leistungsintensive Routinen überprüfen
	\end{itemize}
	\item Bis 07.07.2020:
	\begin{itemize}
		\item Betatest durchführen (mit ca. 30 Teilnehmern bzw. Emulatoren)
		\item Bei Betatest gefundene Bugs fixen
	\end{itemize}
	
\end{itemize}
