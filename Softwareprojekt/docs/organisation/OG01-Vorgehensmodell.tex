\section{Vorgehensmodell}
In diesem Kapitel wird erläutert, für welches Vorgehensmodell wir uns entschieden haben und warum wir diese Entscheidung getroffen haben: \\ 

Für das Projekt: \textit{Smartphone-Client für das Anonymisierungssystem Hydra} haben wir den Unified Process gewählt.\\
Im Folgenden wird mit Ausschlussverfahren aufgezeigt, warum sich dieses Modell am meisten für unsere Anwendung eignet.\\ 

Das Wasserfallmodell kommt für unser Projekt nicht in Frage, da es zwar eine klare Trennung der Phasen gibt, aber phasenübergreifende Änderungen wahrscheinlich sind. Dies geht auf die Tatsache zurück, dass niemand von uns auf Expertise in Android-Entwicklung verweisen kann. Aus diesem Grund ist es nicht auszuschließen, dass es im weiteren Verlauf zu erneuten Änderungen an der Systemarchitektur kommen kann. \\ 

Das agile Vorgehensmodell kommt ebenfalls nicht in Frage, da unser Vorgehen mit viel Dokumentation nicht mehr agil ist. Zudem entwickelt das Fachgebiet \textit{Telematik/Rechnernetze}, der \textit{Technischen Universität Ilmenau}, mit dem Team um \textit{David Schatz},  parallel das Hydra-System. Daher ist es nicht sinnvoll, so schnell wie möglich einen Prototypen zu erstellen. \\ 

Unser iteratives Modell zielt darauf ab, einen Prototyp zu entwickeln, auf dem man vorerst eine, maximal 200 Zeichen lange, Nachrichten von Alice zu Bob übermitteln kann. Anfangs soll dies über nur einen Rendevouz-Punkt geschehen, des Weiteren wird vorerst auf die Onion-Encryption verzichtet. Im weiteren Verlauf implementieren wir kontinuierlich Erweiterungen und Funktionen. In welcher Reihenfolge diese Funktionen implementiert werden, kann im Kapitel \nameref{sec:Meilensteine} nachvollzogen werden. Zu den Produktfunktionen werden, bei ausreichender Zeit, alle Wunschfunktionen, Pflichtenheft Kapitel 4, realisiert.