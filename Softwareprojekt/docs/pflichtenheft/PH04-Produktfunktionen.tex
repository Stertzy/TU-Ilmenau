\section{Produktfunktionen}
In diesem Kapitel werden alle Funktionen beschrieben, die das System ausführen kann. 
Es wird dabei zwischen Pflichtfunktionen (\textbf{/FXXXX/}) und Wunschfunktionen (\textbf{/FXXXXW/}) unterschieden. 
Die Pflichtfunktionen sind entweder notwendig für die Funktionalität des Systems oder dienen der Benutzerfreundlichkeit.
Die Wunschfunktionen hingegen sind optional und nicht für die grundlegende Funktionalität des Systems notwendig.

\subsection{Benutzerfunktionen}
Im Folgenden werden alle Funktionen beschrieben, die ein Benutzer der App aktiv nutzen kann.
\subsubsection{Benutzer-Kennung}
	\begin{description}
		\item[/F0010W/]
			\textit{Passwort setzen:}
			Ein neuer Benutzer muss beim ersten Start der App ein \textbf{Passwort setzen}. 
			Das Passwort muss dabei zweimal angegeben werden, wobei sich diese Angaben nicht unterscheiden dürfen.
		\item[/F0020W/]
			\textit{Anmelden:}
			Der Benutzer muss sich bei jedem App-Start \textbf{anmelden}. 
			Dafür muss er sein \textbf{Passwort} eingeben.
			Voraussetzung dafür ist, dass der Benutzer bereits ein Passwort gesetzt (\textbf{/F0010/}) hat und die Hash-Chain initialisiert ist.
		\item[/F0030W/]
			\textit{Passwort ändern:}
				Der Benutzer kann das \textbf{Passwort ändern}.
				Das neue Passwort muss zweimal angegeben werden, wobei sich diese Angaben nicht unterscheiden dürfen.
		\item[/F0040W/]
			\textit{Registrieren:}
			Der Benutzer kann sich beim \ac{CS} \textbf{registrieren}.
			Zum Registrieren benötigt der Benutzer seine eigene, bzw. private \textbf{E-Mail-Adresse}.
			Der Benutzer kann dabei selbst ein PGP-Schlüsselpaar importieren, falls er dies möchte. 
			Ansonsten wird das System ein Schlüsselpaar erstellen.\\
			Die Registrierung ist erfolgreich, wenn die \textit{E-Mail-Adresse} innerhalb des Systems jeweils eindeutig ist. 
			Die \textit{E-Mail-Adresse} wird auf ihre Gültigkeit geprüft. 
			Mit dem erfolgreichen Abschließen des Registrierungsvorgangs ist der neue Benutzer  dem \ac{CS} bekannt.
	\end{description}
\subsubsection{Kontakte hinzufügen}
Der Benutzer verfügt über ein \textbf{Kontaktbuch}, in dem er andere Benutzer als \textbf{Kontakte speichern} kann. \\
\\
Der Benutzer hat in dem idealen Hydra-System (indem alle Wunschfunktionen implementiert sind) zwei Möglichkeiten, einen Kontakt hinzuzufügen.
	\begin{description}
		\item[/F0110/]
			\textit{Hinzufügen(1):}
			Der Benutzer kann einen anderen Benutzer als \textbf{Kontakt hinzufügen}.
			Dazu müssen sie sich persönlich treffen. 		
			Dabei wird ein gemeinsamer Schlüssel über zwei \textbf{QR-Codes} (einen je Benutzer) ausgetauscht und eine Hash-Chain initialisiert.
		\item[/F0120W/]
		\textit{Hinzufügen(2):}
			Der Benutzer $A$ kann einen anderen Benutzer $B$ als \textbf{Kontakt hinzufügen}. 
			Dafür benötigt Benutzer $A$ die \textbf{E-Mail-Adresse} von Benutzer $B$.
			Im Gegensatz zu \textbf{/F0110/} muss Benutzer $B$ sich im System registriert (\textbf{/F0040W/}) haben.\\
		Zudem führt das System  einen \ac{DHKE} mit einem anderen System (App eines anderen Benutzers) aus. 
		Dabei wird auch die Authentizität anhand des öffentlichen Schlüssels überprüft. Mit diesem Vorgang wird außerdem die Hash-Chain initialisiert.
	\end{description}
	
\subsubsection{Kontaktverwaltung}
	\begin{description}
		\item[/F0210/]
			\textit{Entfernen:}
			Der Benutzer kann einen \textbf{Kontakt entfernen}.
		\item[/F0220/]
			\textit{Bearbeiten:}
			Der Benutzer kann einen \textbf{Kontakt bearbeiten} und somit den \textbf{Nicknamen} des Kontaktes ändern.
		\item[/F0230W/]
			\textit{Blockieren:}
				Der Benutzer kann einen \textbf{Kontakt blockieren}. 
				Dadurch wird jegliche Kommunikation zwischen dem Benutzer und dem Kontakt unterbunden.	
		\item[/F0240W/]
			\textit{Freischalten:} 
			Der Benutzer kann einen \textbf{blockierten Kontakt freischalten}.
			Dadurch können beide Benutzer wieder über den \ac{HC} in Kontakt treten.		
	\end{description}
	
\subsubsection{Chatfunktionen}
	\begin{description}
		\item[/F0310/]
			\textit{Chat starten:}
			Der Benutzer kann einen \textbf{Chat} mit einem Kontakt \textbf{starten}. 
		\item[/F0320/]
			\textit{Textnachricht übermitteln:} 
			Der Benutzer kann in einem Chat eine \textbf{Textnachricht} an einen Kontakt \textbf{übermitteln}.
			Eine Textnachricht besteht dabei maximal aus \textbf{200 ASCII-Zeichen}.
		\item[/F0330/]
			\textit{Chat beenden:}
			Der Benutzer kann den \textbf{Chat} mit einem Kontakt \textbf{beenden}. 
		\item[/F0340/]
			\textit{Chatverlauf löschen:}
			Der Benutzer kann den \textbf{Chatverlauf} eines Chats \textbf{löschen}.
		\item[/F0350W/]
			\textit{(Bild-)Dateien übermitteln:}
			Der Benutzer kann in einem Chat eine beliebige \textbf{(Bild-)Datei} an einen Kontakt \textbf{übermitteln}. 
			Dabei wird es eine Obergrenze der Dateigröße geben, welche erst später im Entwurf festgelegt werden kann.
			Der Benutzer kann den Sendevorgang abbrechen.\\
			Voraussetzung für diese Funktion ist, dass die Funktionen \textbf{/F0610W/}, \textbf{/F0620W/} und \textbf{/F0630W/} implementiert sind.
		\item[/F0360W/]
			\textit{Chateinträge löschen:}
			Der Benutzer kann eine gesendete \textbf{Nachricht} lokal aus dem Chatverlauf \textbf{löschen}.
		\item[/F0370W/]
			\textit{Chat archivieren:}
				Der Benutzer kann einen \textbf{Chatverlauf} lokal \textbf{archivieren}.
		\item[/F0380W/]
			\textit{Chat dearchivieren:}
				Der Benutzer kann einen \textbf{Chatverlauf} lokal \textbf{dearchivieren}.			
	\end{description}

\subsubsection{Personalisierungsfunktionen}
	\begin{description}
		\item[/F0410W/]
			\textit{Design ändern:}
				Der Benutzer kann das \textbf{Design} der Benutzeroberfläche \textbf{ändern}.
				Er kann sich dabei zwischen einem  \glqq  Light-Mode\grqq   $\,$ und einem  \glqq Dark-Mode\grqq $\,$ entscheiden.
		\item[/F0420W/]
			\textit{Hintergrundbild ändern:}
			Der Benutzer kann das \textbf{Hintergrundbild ändern}, welches bei den Chats angezeigt wird.
	\end{description}
	
\subsection{Hydra-Funktionen}
Im Folgenden werden alle Funktionen beschrieben, die das System aufgrund des Hydra-Protokolls ausführen muss.\\
\\
Des Weiteren sei eine \textit{Nachricht} eine Textnachricht oder eine (Bild-)Datei, die der Benutzer an einen Kontakt übermitteln möchte.
	\begin{description}
		\item[/F0510/]
			\textit{Kommunikation mit dem \ac{DS}:}
			Das System \textbf{kommuniziert} jede Epoche mit dem \ac{DS}.
			Dabei erhält es Informationen für die nächsten anstehenden Epochen.
			Diese Information bestehen für je eine Epoche aus:
			\begin{itemize}
				\item
					Alle \textbf{verfügbaren Mixe}, welche folgenden Informationen beinhalten:
					\begin{itemize}
						\item
							IP-Adresse
						\item
							Port
						\item
							Public Key (öffentlicher Teil von \ac{ECDH})
					\end{itemize}
				\item 
					\textbf{Länge des Pfades}
				\item 
					\textbf{Epochennummer und -startzeit}
				\item 
					\textbf{Wartezeit} zwischen den Kommunikationsrunden
				\item
					\textbf{Dauer} einer Kommunikationsrunde
				\item 
					\textbf{Anzahl der Kommunikationsrunden} pro Epoche
			\end{itemize}
		\item[/F0520/]
			\textit{Circuit generieren:} 
			Das System \textbf{generiert} einen \textbf{Circuit} nach den Hydra-Spezifikationen.
		\item[/F0530/]
			\textit{Setup-Packet erstellen:} 
			Das System \textbf{erstellt} ein \textbf{Setup-Packet} für jede Epoche aus \textbf{/F0510/} nach den Hydra-Spezifikationen.
		\item[/F0540/]
			\textit{Setup-Packet versenden:}
			Das System \textbf{versendet} jede Epoche und so früh wie möglich mindestens ein \textbf{Setup-Packet} nach den Hydra-Spezifikationen.
		\item[/F0550/]
			\textit{Nachrichten versenden:} 
			Das System \textbf{puffert und versendet Nachrichten} nach den Hydra-Spezifikationen.
	\end{description}
	
\subsection{Ende-zu-Ende-Funktion}
Im Folgenden werden alle Funktionen aufgelistet, die für die Ende-zu-Ende Übertragung der Nachrichten zuständig sind.
\subsubsection{Fragmentierungsfunktionen}
Die hier aufgelisteten Wunschfunktionen (\textbf{/F0610W/, /F0620W/, /F0630W/}) führen dazu, dass die in \textbf{/F0320/} genannte maximale Größe einer Nachricht von 200 Zeichen aufgehoben wird. 
Dabei wird es eine neue Obergrenze für die maximale Größe einer Nachricht geben, welche aber erst (wie in \textbf{/F0350W/} schon beschrieben) später im Entwurf festgelegt werden kann.
	\begin{description}
		\item[/F0610W/]
			\textit{Nachrichten komprimieren:} 
			Das System \textbf{komprimiert} jede \textbf{Nachricht}.
		\item[/F0620W/]
			\textit{Nachrichten fragmentieren:}
			Das System kann große \textbf{Nachrichten in Fragmente aufteilen}.
		\item[/F0630W/]
			\textit{Nachrichten defragmentieren:}
			Das System kann \textbf{Fragmente}  von Nachrichten \textbf{erkennen} und diese \textbf{zusammenfügen}.
	\end{description}
	
\subsubsection{Kommunikationsfunktionen}
	\begin{description}
		\item[/F0710/]
			\textit{Nachrichten Ende-zu-Ende Verschlüsseln:}
			Das System \textbf{verschlüsselt Nachrichten} mittels einer \ac{E2EE}. Zudem wird ein Integritätsschutz, falls nicht von der \ac{E2EE} gewährleistet, implementiert.
			Somit wird die Vertraulichkeit und Integrität geschützt.
		\item[/F0720/]
			\textit{Zuverlässige Nachrichtenzustellung:}
			Das System \textbf{versucht} eine \textbf{Nachricht zuzustellen}. 
			Nach 48 Stunden erfolgloser Versuche wird das System selbstständig keinen weiteren Versuch unternehmen.
			Der Benutzer wird dann gefragt, ob das System es noch einmal versuchen soll.
		\item[/F0730W/]
			\textit{Push-Benachrichtigung:}
			Das System teilt dem Benutzer mittels einer \textbf{Push-Benachrichtigung} mit, wenn er eine Nachricht von einem anderen Benutzer erhalten hat.

	\end{description}
















