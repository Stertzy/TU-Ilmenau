\section{Benutzeroberfläche}
\textnormal{Im Folgenden werden die Funktionen der Benutzeroberfläche mit \textbf{/BXXXX/} bzw. \textbf{/BXXXXW/} beschrieben.
    Das W markiert analog zu Kapitel 4 Wunschfunktionen.}

\subsection{Startseite}
		\begin{description}
			\item[/B0010W/]
				\textit{Authentisieren:}
					\textnormal{Der Nutzer muss sich beim Start durch Fingerabdruck, Passwort o.ä. authentisieren.}
		\end{description}
\subsection{Hauptseite: Liste aktiver Chats }
		\begin{description}
			\item[/B0110/]
				\textit{Chat erstellen:}
					\textnormal{Button zum Starten eines neuen Chats (Kontakt auswählen in einer Liste mit Suchfunktion).}
			\item[/B0120/]
				\textit{Chat öffnen:}
					\textnormal{Button zum Öffnen eines aktiven Chats.}
			\item[/B0130/]
				\textit{Chat löschen:}
					\textnormal{Button zum Löschen des Chats.}
		\end{description}
\subsection{Seite: Chat}
		\begin{description}
			\item[/B0210/]
				\textit{Nachricht senden:}
					\textnormal{Button zum Senden der geschriebenen (Text-)Nachricht.}
			\item[/B0220/]
				\textit{Chat beenden:}
					\textnormal{Button zum Beenden eines Chats (zurück zur Hauptseite).}
			\item[/B0230W/]
				\textit{(Bild-)Dateien senden:}
					\textnormal{Button öffnet Menü um zu versendende Datei auf dem Endgerät zu suchen.}
		\end{description}
\subsection{Seite: Kontaktverwaltung}
		\begin{description}
			\item[/B0310/]
				\textit{Hinzufügen(1):}
					\textnormal{QR-Code-Scanner zum Hinzufügen anderer Nutzer. Zudem kann dort ein Nickname für den hinzugefügten Nutzer gesetzt werden.}
			\item[/B0320W/]
				\textit{Hinzufügen(2):}
					\textnormal{Eingabefeld für die E-Mail des Nutzers, der hinzugefügt werden soll. Zudem kann dort ein Nickname für den hinzugefügten Nutzer gesetzt werden. Der hinzugefügte Nutzer muss dann die Anfrage bestätigen.}
			\item[/B0330/]
				\textit{Entfernen:}
					\textnormal{Delete-Button zum Löschen von Kontakten.}
			\item[/B0340/]
				\textit{Bearbeiten:}
					\textnormal{Nickname eines Kontaktes ändern.}
			\item[/B0350W/]
				\textit{Blockieren:}
					\textnormal{Button zum Blockieren des jeweiligen Kontakts.}
			\item[/B0360W/]
				\textit{Freischalten:}
					\textnormal{Button, um die Blockierung eines Kontaktes aufzuheben.}
		\end{description}
\subsection{Seite: Registrierung}
			\begin{description}
				\item[/B0410W/]
					\textit{Registrieren(1):}
                        \textnormal{Eingabefeld für die E-Mail-Adresse, mit der sich der Nutzer beim \ac{CS} registrieren will.}
				\item[/B0420W/]	
					\textit{Registrieren(2):}
						\textnormal{Der Nutzer kann auswählen, ob er ein PGP-Schlüsselpaar importieren will oder eines generiert werden soll.}
			\end{description}
\subsection{Seite: Einstellungen}
			\begin{description}
			\item[/B0510W/]
				\textit{Sprache ändern:}
					\textnormal{Durch Betätigen des Buttons wird eine Liste geöffnet, in welcher zwischen allen implementierten Sprachen gewählt werden kann.}
			\item[/B0520W/]
				\textit{Farbthema wechseln:}
					\textnormal{Durch Betätigen des Buttons kann zwischen Dark- und Lightmode gewechselt werden.}
			\end{description}
			






