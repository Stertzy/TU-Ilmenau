\section{Qualitätszielbestimmungen}

\begin{center}
 \begin{tabular}{l|c|c|c|c}
  ~ & sehr wichtig & wichtig & weniger wichtig & unwichtig\\
  \hline \hline
  \textit{Robustheit}~ &  ~ ~ ~ & \textbf{X}~ &  ~ ~ ~ &  ~ ~ ~ \\
  \hline
  \textit{Korrektheit}~ & \textbf{X}~ &  ~ ~ ~ &  ~ ~ ~ &  ~ ~ ~ \\
  \hline
  \textit{Benutzerfreundlichkeit}~ &  ~ ~ ~ & \textbf{X}~ &  ~ ~ ~ &  ~ ~ ~ \\
  \hline
  \textit{Effizienz}~ &  ~ ~ ~ &  ~ ~ ~ & \textbf{X}~ &  ~ ~ ~ \\
  \hline
  \textit{Energieeffizienz}~ &  ~ ~ ~ & \textbf{X}~ &  ~ ~ ~ &  ~ ~ ~ \\
  \hline
  \textit{Portierbarkeit}~ &  ~ ~ ~ & \textbf{X}~ &  ~ ~ ~ &   ~ ~ ~ \\
 \end{tabular}

    \begin{description}
        \item[Robustheit]
            bezeichnet hier die Toleranz gegenüber...
            
\begin{itemize}

            \item ... fehlerhaften Nutzereingaben (passende Fehlerbehandlung),
            \item ... Verlust der Internetverbindung (zuverlässige Übertragung, neu versuchen bei wiederhergestellter Verbindung)
            und 
            \item ... Systemabstürzen (Hydra-Protokoll wird nach Neustart gemäß Spezifikation weitergeführt).
\end{itemize}            


        \item[Korrektheit]
            bezeichnet hier vor allem die Abwesenheit von Fehlern in der Implementierung des Hydra-Protokolls
            und von kryptographischen Mechanismen, die die Sicherheit der Nutzer gefährden könnten.

        \item[Benutzerfreundlichkeit]
            bezeichnet hier vor allem, wie intuitiv die
            Bedienung der graphischen Benutzeroberfläche für die Nutzer der Zielgruppe ist.

        \item[Effizienz]
            bezeichnet hier die Sparsamkeit bezüglich Rechenzeit und Speichernutzung.

        \item[Energieeffizienz]
            bezeichnet hier das Befolgen von best practices bei der Implementierung von Hintergrundprozessen.

        \item[Portierbarkeit]
            bezeichnet hier die Verringerung des Aufwands,
            die Client-App für andere Plattformen (insbesondere iOS) neu zu implementieren.
    \end{description}
\end{center}
