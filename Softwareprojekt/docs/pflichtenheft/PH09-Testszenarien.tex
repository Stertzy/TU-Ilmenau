\section{Globale Testszenarien und -fälle}
	
	Jede Produktfunktion \textbf{/FXXXX/} wird anhand von konkreten Testfällen \textbf{/TXXXX/} getestet. 	
Jede Wunschfunktion \textbf{/FXXXXW/}wird von jeweils \textbf{/TXXXXW/} getestet. 
Voraussetzung dafür ist, dass die entsprechende Wunschfunktion \textbf{/FXXXXW/} implementiert wurde. \\
Die Testszenarien werden von den Qualitätsmanagern, siehe Organisatorisches Dokument, durchgeführt. Dazu verwenden Sie zwei Smartphones, welche den Anforderungen, siehe \nameref{sec:Hardware}, entsprechen.\\
Die dabei verwendeten Namen werden rein zufällig gewählt.

	
	\subsection{Testszenarien für Benutzerfunktionen}

Im Folgenden werden alle Funktionen getestet, die ein Benutzer der App aktiv nutzen kann.

		\subsubsection{Benutzer-Kennung}

			\begin{description}
		\item[/T0010/]
			\textit{Passwort setzen:} 
			\textnormal{Bob startet zum ersten Mal den \ac{HC}. 
			Nun gibt er zwei mal in Folge sein Passwort \textit{//TestPW01} korrekt ein. }

		\item[/T0020/] 
			\textit{Anmelden:} 
			\textnormal{Alice startet den \ac{HC} zum wiederholten Male. 
			Um Zugriff auf den \ac{HC} zu gelangen, muss sie das Passwort, was sie in \textbf{/T0010/} gesetzt hat, korrekt eingeben.}
			
			
		\item[/T0030/]
			\textit{Passwort ändern:}
			\textnormal{Bob will sein Passwort von \textit{Test12!!} zu \textit{Test34!!} ändern. 
			Um dies umzusetzen gibt er zuerst das alte Passwort einmal korrekt ein und anschließend das neue Passwort zweimal korrekt ein.}
			
		\item[/T0040W/]
			\textit{Registrieren:}
			\textnormal{Bob möchte sich beim \ac{CS} registrieren. 
			Dafür benutzt er seine private E-Mail-Adresse \textit{
			bob@test.de}. }
			
			
\subsubsection{Kontakte hinzufügen}

			
		\item[/T0110/]
			\textit{Hinzufügen(1):} 
			\textnormal{Alice und Bob wollen im \ac{HC} kommunizieren. 
			Dazu treffen sie sich persönlich. Die Kontaktdaten werden mit einem QR-Code Scan ausgetauscht. 
			  }			
			
		\item[/T0120W/]
			\textit{Hinzufügen(2):} 
			\textnormal{Alice und Bob möchten im \ac{HC} kommunizieren. 
			Dafür benötigt einer von beiden, hier Alice, die E-Mail-Adresse von Bob: \textit{bob@test.de}. 
			Dies klappt ohne Probleme, da Bob sich in \textbf{/T0030W/} beim \ac{CS} registriert hat .}
			
\subsubsection{Kontaktverwaltung}
			
		\item[/T0210/]
			\textit{Entfernen:}
			\textnormal{Alice entfernt Bob aus ihren Kontakten.}
			
		\item[/T0220/]
			\textit{Bearbeiten:}
				\textnormal{Alice ändert den Nickname von Bob.}
				
		\item[/T0230W/]
			\textit{Blockieren:}
				\textnormal{Alice blockiert Bob. }
				
				\item[/T0240W/]
			\textit{Freischalten:}
				\textnormal{Alice schaltet Bob wieder frei. }

\subsubsection{Chatfunktion}

		\item[/T0310-T0320/]
			\textit{Chat starten und Textnachricht übermitteln:}
				\textnormal{Um den Chat zu starten, sendet Alice eine zehn Zeichen lange Nachricht an Bob. \\ Anschließend schickt Bob eine 200 Zeichen lange Nachricht and Alice. }
				
		
		\item[/T0330/]
			\textit{Chat beenden:}
			\textnormal{Alice beendet den Chat mit Bob.}		
			
					\item[/T0340/]
			\textit{Chatverlauf löschen:}
			\textnormal{Alice löscht den Chatverlauf mit Bob.}			
			
		\item[/T0350W/]
			\textit{(Bild-)Dateien übermitteln:}
			\textnormal{Bob schickt Alice eine 50KB große Bilddatei.} 
			
		\item[/T0360W/]
			\textit{Textnachricht löschen:}
			\textnormal{Alice löscht eine Nachricht im Chat mit Bob.} 
			
				\item[/T0370W/]
			\textit{Chat archivieren:}
			\textnormal{Alice archiviert den Chat mit Bob.} 
			
						\item[/T0380W/]
			\textit{Chat dearchivieren:}
			\textnormal{Alice dearchiviert den Chat mit Bob.} 
			
			
		\subsubsection{Personalisierungsfunktionen}
		
		\item[/T0410W/]
			\textit{Design ändern:}
			\textnormal{Alice stellt das Design vom \textit{Dark-Mode} in den \textit{Light-Mode} .} 
			
			
		\item[/T0420W/]
			\textit{Hintergrundbild ändern:}
			\textnormal{Alice ändert das Hintergrundbild für Chats.} 
			
			
			
	\subsection{Testszenarien für Hydra-Funktionen}	

Im Folgenden werden alle Funktionen getestet, die das System aufgrund des Hydra- Protokolls ausführen muss.
		
			\item[/T0510-T0550/]
				\textit{Setup Vorgang:}
				\textnormal{Um die korrekte Funktionalität ihres \ac{HC} zu testen, schickt Alice eine 150 Zeichen lange Nachricht, sowie eine 100KB große Bilddatei an Bob. Wenn beide korrekt ankommen, weiß sie, dass das Setup korrekt ausgeführt wurde.}
				
				\subsection{Testszenarien der Ende-zu-Ende-Funktion}
				Im Folgenden werden alle Funktionen getestet, die für die Ende-zu-Ende Übertragung der
 Nachrichten zuständig sind.
 
		\subsubsection{Fragmentierungsfunktionen}
			\item[/T0610W-T0630W/]
				\textit{Fragmentierungsfunktion:}
				\textnormal{Bob sendet eine 5000 Zeichen lange Textnachricht an Alice.}
				
		\subsubsection{Kommunikationsfunktionen}	
		
		
		
			\item[/T0720/]
				\textit{Zuverlässige Nachrichtenzustellung:}
				\textnormal{Alice möchte kontrollieren, ob Nachrichten zuverlässig zugestellt werden. Dazu unterdrückt sie alle Acknowledgements, die ihr \ac{HC} versenden würde.}
				
			\item[/T0730W/]
				\textit{Push-Benachrichtigung:}
				\textnormal{Thomas Müller sendet eine Textnachricht an Tina Musterfrau.}
				
				
	




			
			
	\end{description}

