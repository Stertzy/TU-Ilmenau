\section{Produkteinsatz}
\textnormal{Im folgenden wird dargelegt, was der \ac{HC} ist, an wen er sich richtet und wozu er dient.}

\subsection{Anwendungsbereich}

\textnormal{Der \ac{HC} ist die Nutzer-Applikation des anonymisierten und mittels einer \ac{E2EE} geschützten Instant-Messenging Systems \textit{Hydra}, wie es im Paper \textit{Hydra: Practical Metadata Security for Contact Discovery, Instant Messaging and Dialing  (unveröffentlichte Version) von David Schatz, Michael Rossberg und Günter Schäfer } dargelegt ist. 
Er soll von Personen im Alltag für private, wie auch berufliche Angelegenheiten genutzt werden können. 
Dabei bietet \textit{Hydra} starke Anonymität, Integrität und Vertraulichkeit.
}


\subsection{Zielgruppe} 


\textbf{Kernzielgruppe:} \\
 
Unsere Kernzielgruppe sind Privatpersonen, die nicht nur bereits Instant-Messenger nutzen, sondern sich zudem den Sicherheitsrisiken anderer Messenger bewusst sind und Alternativen suchen (viel Wert auf Anonymität, Integrität und Vertraulichkeit legen), aber eine Ende-zu-Ende Latenz von 10 bis 30 Sekunden tolerieren können.
\newline\linebreak
\textbf{Weitere Merkmale der Kernzielgruppe:}
\begin{itemize}
	\item Demographische Merkmale
	\begin{itemize}
		\item Alter: 16+
		\item Wohnort: Gebiete mit Internetanbindung
	 \end{itemize}
	 \item Soziologische Merkmale
	 \begin{itemize}
	 	\item Bildungsstand: eher gehoben (Bewusstsein von Sicherheitsrisiken anderer Messenger)
	 	\item Einstellung: Halten Anonymität und Integrität für wichtig 
	 	
	 \end{itemize}	 	
\end{itemize}



\textbf{Ausgedehnte Zielgruppe:}\\

Die ausgedehnte Zielgruppe beläuft sich hingegen auf sämtliche Privatpersonen, die bereits Erfahrung mit Messengern haben und ein mobiles Endgerät besitzen, auf welchem wenigstens Android 8.0 läuft.


Weitere Merkmale der Zielgruppe unterscheiden sich kaum von denen der Kernzielgruppe, mit Ausnahme von:
\begin{itemize}
		\item Alter: 9+
		\item Bildungsstand: es genügt die Vertrautheit mit Smartphone-Apps im Allgemeinen
		\item Einstellung: irrelevant, das heißt Anonymität und Sicherheit steht nicht unbedingt im Fokus
\end{itemize}


\subsection{Betriebsbedingungen}
\begin{itemize}
	\item Der \ac{HC} läuft unter vollständiger Funktionalität der anderen Komponenten des Hydra Systems korrekt
	\item Betriebsdauer: täglich, bis zu 24 Stunden
	\item Muss ggf. regelmäßig aktualisiert werden, um kompatibel mit Android, als auch dem Hydra-Gesamtsystem zu bleiben

\end{itemize}
Dieses System soll sich hinsichtlich der Betriebsbedingungen von anderen Android Applikationen (und insbesondere Messengern) nicht wesentlich unterscheiden (mit Ausnahme der höheren Latenz).